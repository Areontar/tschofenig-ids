%\documentclass[letterpaper, 10pt]{sigcomm-alternate}

\documentclass[peerreview, a4paper, draft, 7pt]{IEEEtran}
%\documentclass[a4paper, 10pt]{IEEEtran}
%\documentclass{scrreprt}
\usepackage{times}
%\usepackage{color}
% \usepackage{caption2}
\usepackage{subfigure}
\usepackage{graphicx}
\usepackage{colortbl}
\usepackage[dvipsnames]{xcolor}
% \usepackage{ulem}

\pagenumbering{arabic}

\ifx\pdfoutput\undefined
\usepackage[pdfpagemode=none, pdfstartview=FitH, colorlinks=true, urlcolor=black, linkcolor=black, citecolor=black, letterpaper]{hyperref}
\else
\usepackage[pdftex, pdfpagemode=none, pdfstartview=FitH, colorlinks=true, urlcolor=black, linkcolor=black, citecolor=black, letterpaper, pdftex]{hyperref}
\fi

%% Editorial work
% \newcommand{\purge}[1]{\textcolor{red}{\sout{#1}}}
% \newcommand{\add}[1]{\textcolor{blue}{#1}}

%%% End of editorial work.


 
\renewcommand{\em}[1]{\textit{#1}}
\begin{document}

\title{Standardizing the Next Generation Public Key Infrastructure \\ \large Incremental Improvements or Re-Design?}

\author{\authorblockN{Hannes Tschofenig\authorrefmark{1}, Tobias Gondrom\authorrefmark{2}\\}
\authorblockA{\authorrefmark{1}Nokia Siemens Networks, 
Email: Hannes.Tschofenig@nsn.com\\}
\authorblockA{\authorrefmark{2}Thames Stanley, 
Email: tobias.gondrom@gondrom.org\\}
\thanks{\textsc{
Position paper for the 'NIST Workshop on Improving Trust in the Online Marketplace', April 10-11, Gaithersburg, US. This paper represents the views of the authors and does not reflect the consensus of the Internet Engineering Task Force (IETF), of any single IETF working group, or of the Internet Architecture Board (IAB). Referenced documents may, however, reflect IETF consensus.}}
}

\date{\today}

\maketitle

\section{Extended Abstract}

High-profile data breaches and security incidents on the Web are gaining increasing attention from the public, the press, and governments. A few examples may illustrate the problems: DigiNotar, a Dutch certificate authority, had a security breach \cite{DigiNotar} and in the same year a Comodo affiliate was compromised \cite{comodo}. Both cases lead to fraudulent issue of certificates and raise questions regarding the strength of the Public Key Infrastructure (PKI) used by Web applications today. Also in 2011 LulzSec, a hacker group, claimed responsibility for several attacks, including the compromise of user accounts from Sony. Notifications about millions of stolen user accounts became common these days. 

These developments have led to a number of activities to improve the security of the Web platform since the Web is in the front-line of these attacks. The IETF had started various standardization activities in response to these developments\footnote{While the title of the workshop is very generic our description focuses on activities related to alternatives to and improvement for the Web PKI. Note that Public Key Infrastructure is not only used for the Web even though the Web infrastructure is the most visible application platform today. Other public key infrastructures may operate differently and may suffer from fewer or at least different problems. There are also problems related to improving on-line trust, such as the user authentication infrastructure, that are worth paying attention to.}. We will mention a few of those below and ask ourselves the question whether further work is needed. 

The Web platform uses Transport Layer Security (TLS \cite{rfc5246}) and relies heavily on the PKI \cite{rfc5280} for it's security. Consequently, a lot of attention has been paid on improving the security of TLS. Many valuable TLS extensions have been defined to offer additional features (e.g., cryptographic algorithm support, new credentials types). Unfortunately, the operational reality had not received the same attention. The PKI model assumed that certificate revocation lists (CRLs) or the Online Certificate Status Protocol (OCSP) \cite{rfc2560} are used but browser manufacturers do not seem to believe strongly in OCSP and CRLs or their constant and fully reliable availability and therefore they are not enabled per-default. The Certification Authority/Browser Forum \cite{CABrowserForum}, an organization of  certification authorities and browser vendors, is largely responsible for today's operational practices of the Web PKI but acted prior to their organizational reform early 2012 operating behind closed doors. The certificate signing practices (e.g., the existence of certificates with unqualified names, IP addresses and other artifacts) and the lack of accountability of CAs can be attributed to the policies in the CA/Browser Forum. The most important side-effect of the PKI model and the Web PKI, however, is the large number of trust anchors that can be found in today's browsers. While end users are at least theoretically in the position to modify the trust anchors, the lack of knowledge and the ability of browsers to outsource path validation to external companies makes it almost impossible for end users to grasp the level of complexity and to judge who their browser trusts. The 2011 incidents have also shown that a single compromised CA can issue certificates for any Internet site since the name-to-key binding is not enforced in the PKI model. Some researchers and security experts call this a "design flaw", that a single  weak entity in a very large number of hundreds of trusted CAs poses risks to the whole system and all users of the Web PKI. This situation also leads in game theoretic terms to disincentives for investments by each individual CA to improve their individual security posture. Although the IETF had even standardized a protocol for dynamically updating trust anchors (see TAMP \cite{rfc5934}) it has not been widely used. Deciding which CA to trust and how many is inherently difficult since different stakeholders (e.g., end users, Website providers, browser vendors, CAs) have very opposing views. 

The IETF had approached the problem in three directions: 
\begin{enumerate}
\item Develop an alternative to the PKI model\footnote{Note that there are also entirely different approaches for distributed authentication. For example, the Application Bridging for Federated Access Beyond Web (ABFAB) working group \cite{draft-ietf-abfab-arch} focuses on a model that is closer to the AAA architecture, which is widely used for network access authentication.} in the form of DNS-Based Authentication of Named Entities (DANE) \cite{rfc6698}, which re-uses the DNS to create the name to key binding. It does, however, rely on DNSSEC for it's operation and will therefore be a mid-to-long-term solution. 
\item Standardize shorter-term solutions in the Web Security (WEBSEC) \cite {WEBSEC} working group, such as the  Public Key Pinning Extension for HTTP \cite{draft-ietf-websec-key-pinning}, HTTP Strict Transport Layer Security (HSTS) \cite{rfc6797}, TLS Channel Bindings \cite{rfc5929}, and alike. 
\item Document how the Web PKI currently works. The working group formation of the working group is in progress \cite{WPKOPS}. The EFF SSL Observatory \cite{ssl-observatory} may provide valuable input for this analysis. 
\end{enumerate}

In addition, the Certificate Transparency proposal \cite{draft-laurie-pki-sunlight}, which describes an alternative to the Web PKI model, has been submitted to the IETF for publication as an RFC. It will serve as an experiment. as the authors describe it. Similarly to EFF's Sovereign Keys \cite{sovereign-keys} the existence of an append-only log with all CA-issued certificates is assumed. 

While the standardization work is progressing at a satisfactory speed, challenges remain. There is no common agreement of the design constraints and the types of threats that are supposed to be mitigated. The threat landscape constantly evolving and an agreement about what threats need to be address does not exist. While the authors of individual approaches have a strong view of how the future CA system should look like, there are subtle but significant differences between the existing proposals. For example, EFF's Sovereign Keys and the Certificate Transparency solutions do not want to place the same level of trust on the DNS as DANE does since the top-level domain operator may be under control of the attacker. Providing security for certain scenarios is in practical terms also extremely hard. For example, designing solutions that work with the coffee shop Internet access, Internet access via captive portal, firewalled enterprise environments , etc. increase the complexity of solutions considerably. Although the Sovereign Keys and the Certificate Transparency proposals are seen as an experiment it is not clear what the success criteria will be. Experiments are important but need to be refined since transforming the Web platform to work over an experimental security infrastructure, which may increase systemic dependencies or may otherwise be fragile, could pose risks as well.

The authors would like to see the formation of an IETF or an IRTF working group\footnote{A few years ago the 'Public Key Next-Generation Research Group (PKNG)' \cite{PKNG} was created to discuss alternative PKI models but it had to be closed due to lack of interest.} to discuss the proposed experiments to reach a better understanding of the design constraints and goals that should be accomplished. We believe it is important to involve a broad range of stakeholders; it will be a prerequisite for the success. 

\bibliographystyle{IEEEtran}
% \bibliographystyle{acmtrans}
\bibliography{paper}
\end{document}
